\documentclass[xcolor={table}]{beamer}

\usetheme{Dragonfly}

\title{Introduction to Haskell\\\& some category theory}
\subtitle{Wellington Functional Programming}
\projectcode{26 March 2015}
\author{Finlay Thompson}
\date{}

\AtBeginSection[]{
  \begin{frame}
  \vfill
  \centering
  \Huge
    \insertsectionhead\par%
  \vfill
  \end{frame}
}

\begin{document}

\titleslide

\section{Introduction}

\begin{frame}{}{}

    \begin{columns}
    \column{0.5\textwidth}
    {\Large Haskell is strange ? }

    \pause\bigskip
    A functional programming language

    \pause\bigskip
    With lazy evaluation
    
    \pause\bigskip
    Pure, with no side effects 
    
    \pause\bigskip
    Fairly old\pause, fairly odd

    \column{0.5\textwidth}
    \includegraphics[width=0.9\textwidth]{images/tron-guy.jpg}
        
    \end{columns}

\end{frame}

\begin{frame}{}{}

    \begin{columns}
    \column{0.5\textwidth}
    {\Large Haskell is hard ?}

    \pause\bigskip
    My program won't compile,\\ and I don't know why ?

    \pause\bigskip
    The tutorials online are confusing.

    \pause\bigskip
    Oh god, I am reading math !

    \pause\bigskip
    Huh ?

    \column{0.5\textwidth}
    \includegraphics[width=0.9\textwidth]{images/content-proxy.png}
        
    \end{columns}

\end{frame}

\begin{frame}{}{}

    \begin{columns}
    \column{0.5\textwidth}
    {\Large Haskell is impractical ? }

    \pause\bigskip
    Strong type system gets in the way

    \pause\bigskip
    Hard to install, and find good libraries

    \pause\bigskip
    Impossible to find other developers

    \pause

    \column{0.5\textwidth}
    \includegraphics[width=0.9\textwidth]{images/ducati-muletto.jpg}
        
    \end{columns}

\end{frame}

\begin{frame}{}{}

    {\Large But, Haskell is awesome! }

    \pause
    Is Haskell weird ?\pause
    \begin{itemize}
        \item No, just different. Its the other languages that are weird.
    \end{itemize}

    \pause
    Is Haskell hard ?\pause 
    \begin{itemize}
        \item No, it makes you think differently, which is good.
    \end{itemize}

    \pause
    Is Haskell impractical ?\pause 
    \begin{itemize}
        \item Hackage has thousands of libraries
        \item Haskell is fast, and getting faster
    \end{itemize}

\end{frame}

\begin{frame}{}{}

    \centering
    {\Large To learn Haskell, \\it helps to learn a little category theory. }
    \par\bigskip\pause

    Actually, I reckon you already know category theory!


\end{frame}

\section{Anatomy of a function}

\begin{frame}{}{}

    \centering
    \includegraphics[width=0.9\textwidth]{images/python-code-01.png}

\end{frame}

\begin{frame}{}{}

    \centering
    \includegraphics[width=0.9\textwidth]{images/python-code-02.png}

\end{frame}

\begin{frame}{}{}

    \centering
    \includegraphics[width=0.9\textwidth]{images/python-code-03.png}

\end{frame}

\begin{frame}{}{}

    \centering
    \includegraphics[width=0.9\textwidth]{images/python-code-04.png}

\end{frame}

\begin{frame}{}{}

    \centering
    \includegraphics[width=0.9\textwidth]{images/python-code-05.png}

\end{frame}


\begin{frame}{}{}

    \centering
    \includegraphics[width=0.9\textwidth]{images/haskell-code-01.png}

\end{frame}

\begin{frame}{}{}

    \centering
    \includegraphics[width=0.9\textwidth]{images/haskell-code-02.png}

\end{frame}

\begin{frame}{}{}

    \centering
    \includegraphics[width=0.9\textwidth]{images/haskell-code-03.png}

\end{frame}

\begin{frame}{}{}

    \centering
    \includegraphics[width=0.9\textwidth]{images/haskell-code-04.png}

\end{frame}

\begin{frame}{}{}

    \centering
    \includegraphics[width=0.9\textwidth]{images/haskell-code-05.png}

\end{frame}


\section{A little category theory}

\begin{frame}{}{}

    \centering
    \only<1>{\includegraphics[width=0.7\textwidth]{images/two-dots-01.png}}

    \only<2>{\includegraphics[width=0.7\textwidth]{images/two-dots-02.png}}

\end{frame}

\begin{frame}{}{}

    {\Large Category theory is all about arrows. }

    \pause\bigskip
    Need to define what the \textbf{objects} are.

    \pause\bigskip
    Between two objects, there are \textbf{arrows}. 

    \pause\bigskip
    There are some rules, more on that later.

\end{frame}

\begin{frame}{}{}

    {\Large Programming involves defining arrows.}

    \pause\bigskip
    The objects in the programming category are \textbf{types}. Types are sets of values.

    \pause\bigskip
    Writing functions in a programming language involves defining arrows between data types. 

    \pause\bigskip
    Haskell emphasises category theory aspect of programming.

\end{frame}

\begin{frame}{}{}

    \emph{Warning, philosophical musing ahead...}

    \pause\bigskip
    Category theory is \textbf{about} the common patterns that emerge when we consider diagrams of objects and arrows.

    \pause\bigskip
    A category defines some kind of objects, and the way we can transform these objects into each other. It is a very general concept, and so almost completely vacuous.

    \pause\bigskip
    As is often the case with mathematical concepts, there is nothing more than the definitions.

\end{frame}

\begin{frame}{}{}

    \begin{columns}
    \column{0.75\textwidth}
    
    \begin{definition}
        A category $C$ consists of
        \begin{itemize}
            \item a class of objects $Obj(C)$, 
            \item $\forall X,Y\in Obj(C)$, $\exists\ $ a class of arrows $C(X,Y)$.
            \item $\forall f:X\to Y$, and $g:Y \to Z$, $\exists\ g\circ f:X \to Z$.
        \end{itemize}
        Such that
        \begin{itemize}
            \item $\forall X\in Obj(C)$, $\exists\ id_X: X \to X$,
            \item $\forall f:X\to Y$, then 
                $$ f \circ id_X = f = id_Y \circ f $$
            \item $\forall f:X\to Y$, $g:Y \to Z$, and $h:Z \to W$, then
                $$ (h \circ g) \circ f = h \circ (g \circ f) $$
        \end{itemize}
    \end{definition}


    \column{0.25\textwidth}
    \includegraphics[width=\textwidth]{images/Commutative_diagram_for_morphism.png}
    \end{columns}

\end{frame}

\section{Programming patterns}

\begin{frame}{}{}


\end{frame}

\section{Functors}

\begin{frame}{}{}


\end{frame}


\section{Monads}

\begin{frame}{}{}


\end{frame}



\end{document}
